
% Load packages: making changes here can cause errors.
\documentclass{article}                 % Define document class. Shouldn't change.
\usepackage{booktabs}       
\usepackage{tabularx}   
\usepackage{import}                     % This package allows us to import files. 
\usepackage{multirow}
\usepackage{adjustbox}                  % This package allows you to adapt table and figure sizes to fit the page and is required by iebaltab
\usepackage{geometry}
\usepackage{longtable}
\usepackage{subcaption}                 % This packages is used to create subfigures
\usepackage{enumitem}
\usepackage{array}
\usepackage{hyperref}
\usepackage{setspace}    
\usepackage{float}    
\doublespacing                          % Uncomment to use double spacing
\usepackage{indentfirst}	            % Indents the fist paragraph of each section
\usepackage{parskip}                    % This packages sets the spacing between two paragraphs
\setlength{\parskip}{.5\baselineskip}   % Define spacing between two paragraphs
\usepackage{pdflscape}
\usepackage{afterpage}
\usepackage[usenames, dvipsnames]{color}
\usepackage{graphicx}
\onehalfspacing
\usepackage{setspace}
\usepackage[english]{babel}
\usepackage{pdfpages}
\usepackage{color}
\usepackage[dvipsnames]{xcolor}
\usepackage{pagecolor}
\usepackage{sectsty}


\title{\color{Black} Who Staffs the Health Facilities in Sub-Saharan Africa: Evidence from the Service Delivery Indicator (SDI) surveys in 11 Countries}
\author{\color{Black} Roberta Gatti, Jishnu Das, Kathryn  Andrews, \\
	Ruben Conner,  Jigyasa Sharma, Andres Yi Chang, and Michael Orevba}


\begin{document}
	
	\maketitle        
	\sectionfont{\color{Black}}
	

	\begin{abstract}
		Primary care is at the center of improving population access to healthcare and yet there are limited studies on medical provider demographics in Sub-Saharan Africa. The World Bank’s Service Indicators (SDI) surveys have attempted to fill this gap. Using SDI surveys of 9,697 health facilities and 87,153 healthcare providers in 11 Sub-Saharan countries, we investigate the demographic makeup of healthcare providers to better understand the average citizen’s experience of primary care in Sub-Saharan Africa. The majority of the health facilities in the sample are health posts, which are predominately located in rural areas. Female providers comprise a majority of the staff regardless of health facility type. In terms of provider cadre, patients are most likely to be taken care of by a nurse or paramedical staff member as these providers are the most present at health facilities daily. Moreover, rural health facilities have a lower share of doctors when compared to their urban counterparts. The SDI surveys offer an opportunity to examine carefully who are the healthcare providers that staff the health facilities in these 11 countries, which is an essential first step towards primary care reform.
	\end{abstract}
	
	\vspace{40mm}
	
	\section{Introduction}
	
	[Jishnu and Ruben will write up this section]
	
	\section{Data}
	
	The Service Delivery Indicator  (SDI) health surveys have been implemented for over ten years across 13 countries in sub-Saharan Africa and this paper presents data combined from across this period. The resulting datasets for this paper includes information on 11 countries, covering 9,712 health facilities and  87,153 health providers. The data includes results from the following country surveys: Guinea Bissau (2018), Kenya (2012),  Kenya (2018), Madagascar (2016), Malawi (2019),  Mozambique (2014), Niger (2015), Nigeria (2013), Sierra Leone (2018), Tanzania (2014), Tanzania (2016), Togo (2013) and Uganda (2013).
	
	For each country, the Ministry of Health provided a complete listing of health facilities that offered primary care services. The sample of facilities were selected from this listing of health facilities. The list included facilities operated by private entities or non-governmental organizations (NGOs), and facilities at all levels of care, including hospitals, health clinics and health posts (or the national equivalent). Hospitals, health centers, and health posts (or national equivalent) that provided primary care services were included in the sample selection. The sample selection was stratified by rural/urban location and by facility-type. All surveys were designed to be nationally representative, except for Nigeria, where data was collected in 12 of 36 states due to logistical constraints and Kenya, where data was representative at the county-level. In some results, GDP per capita  estimates were used to show the correlation with the availability of a doctor. The estimates of GDP per capita (based on purchasing power parity) come from World Bank Open Data for the year 2011 in each country. \cite{wbgdp} 
	
	While most of the health facilities in the SDI datasets are publicly operated, the datasets do  include private for-profit facilities (10 percent of sample), faith-based non-profit facilities (7.6 percent), NGO facilities (1.8 percent), and community-run organizations (0.5 percent). The surveys also  collects information on the full roster of staff members in a health facility during the initial visit and asks providers to list information about other providers in the health facility. Table \ref{prov_des} displays some summary characteristics of providers across the 11 countries that were interviewed in the SDI surveys.  As one can see, a majority of the providers are women and health posts are the the main type of  health facilities across all countries.  
	
	\section{Results}
	
	The results section is divided into two subsections. First, we demonstrate that providers are generally younger ranging from 30-40 years old, even when provider cadre is take into account. We show that it is mostly young nurses and para-professional staff that patients interact with then they enter their local health facility. Then, we examine the availability of doctors at health facilities in each country and how that measure varies from urban to rural communities. We also indicate that the presence of doctors is highly correlated with the GDP  within each country, with countries that have high povery rates having a lower share of clinics that have a doctor. 
	
	\subsection{The Demographic Makeup of Providers in Sub-Saharan Africa}
	
	Figure \ref{age_prov} displays the age of providers by cadre in interval bandwidths of 10 years. As one can see, regardless of medical profession, most of the health providers are young between the ages of 30-40 years old. Approximately 40\% of the doctors in this sample are between the ages of 30-40 which is strikingly young compared to average age of doctors in the other parts of the world. Around 33\% and 37\% of nurses and para-professional staff, respectively, are between the ages of 30-40 years old. Overall, a majority of the health providers are under the age of 40 and this is a reflection of how young the adult population is in Sub-Saharan Africa. Furthermore, more we know from Table \ref{prov_des} that most of these young providers are predominately women. When a patient walks into a health facility seeking care in these 11 countries, that patient is mostly likely to be taken care of by a young female nurse or pare-professional staff member.
			
	\subsection{The Availability of Doctors}			
	
	The availability of a doctor is defined as  the number of clinics that have at least one medical doctor per administrative level in the country. The presence of a doctor in a clinic could have profound impacts on  whether a patient with a serious illness is accurately diagnosed and given the appropriate treatment and medication. Figure \ref{fem_doc} depicts the fraction of female doctors in each country by age. As one can see, Kenya in 2012 and Madagascar have the highest fraction of female doctors and the share of female doctor does not vary much as age increases. Nigeria has the lowest fraction of female doctors and it does not increase as age goes up. The average of all countries hovers around 30\% and fluctuates slightly as age goes but not by much. This illustrates that male doctors are still the largest share of medical doctors across these countries even then age is taken into account. On average, patients in this cohort of countries are most likely to interact with a young male doctor if a doctor is present at their nearest clinic.  
	
	Figure \ref{doc} depicts the of share of clinics that have a doctor per administrative level by GDP at the \$1.9 in each country. The share of clinics that have a doctor falls as the poverty rate at the \$1.9 increases and this is reinforced by the downward slope in the graph.
	
	
	
	\subsection{Limitations of the Sample}	
	
	This section will describes details of the SDI surveys and their shortcomings which ultimately limit the scope of what can be inferred about providers and health facilities in the countries that were surveyed. As mentioned earlier, all surveys were designed to be nationally representative, except for Nigeria, where data was collected in 12 of 36 states due to logistical constraints and Kenya, where data was representative at the county-level. The statistics reported in this paper  are based on using survey weights, which were created based on the inverse probability of being sampled. However, sample weights were unavailable for Guinea Bissau, Malawi, and Mozambique, so the unweighted results are reported. 
	
	Moreover, the SDI surveys are designed to allow flexibility in adapting to country-specific challenges. The interaction between national and cross-country experience  have changed both what some SDI surveys measure and how they measure it. Likewise, locations were discarded from the sampling process due to security concerns or other logistical restrictions. In  Mozambique, the sample was reduced from 300 health facilities originally to 204 facilities, due to logistical and financial problems. However, these exclusions did not affect the representativeness of results at the national level. In  Sierra Leone, all hospitals and health centers were selected within stratum, while other facility types like clinics and health posts were randomly sampled.
	
	It should be noted that provider and facility characteristics can be influenced by factors outside of the facility, including  demographic characteristics of the catchment population, and  the public sector governance. Coupled with the expected differences in survey implementation across different settings, can limit  comparability of the survey results. Some of these differences can be mitigated, by using survey weights, when available, to correct for differences in samples across countries. However, certain differences, cannot be corrected analytically. Therefore, the differences between countries should be interpreted with caution, including as a reflection of differences in survey design and implementation.
	
	Overall, sampling procedures are properly planned ahead of data collection activities. Contextual decisions to non-randomly select or eliminate certain geographic locations  are taken ex-ante and in manners that do not compromise rigor and representativeness of results, as well as comparability within objective stratum and across countries. 
	
	
	\section{Discussion}
	
	[Jishnu and Ruben will write up this section]

	\newpage

	\bibliographystyle{unsrt}
	\bibliography{sample} 
	
	\newpage
	
			\begin{table}[H]
				\centering
				\caption{Provider Characteristics}
						\begin{adjustbox}{max width=1\textwidth}
							\def\sym#1{\ifmmode^{#1}\else\(^{#1}\)\fi}
\begin{tabular}{l*{7}{c}}
\hline\hline
&\multicolumn{2}{c}{Gender} &\multicolumn{3}{c}{Facility Level}    \\\cmidrule(lr){2-3}\cmidrule(lr){4-6}
&\multicolumn{1}{c}{Female}&\multicolumn{1}{c}{Male}&\multicolumn{1}{c}{Hospital}&\multicolumn{1}{c}{Health Center}&\multicolumn{1}{c}{Health Post}&\multicolumn{1}{c}{Total Providers}&\\
\hline
Guinea Bissau&         {61(47\%)}&        {69(53\%)}&    {6(5\%)}&               {124(95\%)}&             {0(0\%)}&             {130}\\
Kenya 2012&            {1922(61\%)}&        {1209(39\%)}&    {1389(44\%)}&               {1314(42\%)}&             {435(14\%)}&             {3138}\\
Kenya 2018&        {13697(56\%)}&    {10706(44\%)}&    {10381(43\%)}&               {5971(24\%)}&             {8052(33\%)}&             {24404}\\
Madagascar&        {1332(61\%)}&    {868(39\%)}&    {517(24\%)}&               {1522(69\%)}&             {161(7\%)}&             {2200}\\
Mozambique&        {1405(47\%)}&    {1567(53\%)}&    {1733(58\%)}&               {186(6\%)}&             {1053(35\%)}&             {2972}\\
Malawi&                {5956(46\%)}&        {6933(54\%)}&    {3769(28\%)}&               {8784(66\%)}&             {722(5\%)}&             {13275}\\
Niger&                 {921(69\%)}&        {406(31\%)}&    {606(46\%)}&               {416(31\%)}&             {309(23\%)}&             {1331}\\
Nigeria&               {14042(66\%)}&        {7151(34\%)}&    {9463(44\%)}&               {10654(50\%)}&             {1201(6\%)}&             {21318}\\
Sierra Leone&      {3019(60\%)}&    {2036(40\%)}&    {1706(34\%)}&               {1190(24\%)}&             {2159(43\%)}&             {5055}\\
Togo&                  {695(51\%)}&  {669(49\%)}&        {360(26\%)}&             {564(41\%)}&             {440(32\%)}&             {1364}\\
Tanzania 2014&     {2881(65\%)}&  {1578(35\%)}&    {1215(27\%)}&             {1701(38\%)}&             {1543(35\%)}&             {4459}\\
Tanzania 2016&     {3402(66\%)}&  {1734(34\%)}&    {1244(24\%)}&             {2216(43\%)}&             {1700(33\%)}&             {5160}\\
Uganda&                {1483(63\%)}&  {859(37\%)}&        {114(5\%)}&             {1359(58\%)}&             {874(37\%)}&             {2347}\\
\hline
Total&                 {50816}&                               {35785}&                       {32503}&                               {36001}&                               {18649}&                       {87153}\\
\hline\hline
\multicolumn{7}{l}{\footnotesize Notes: Gender is reported among the providers in the sample of 87,153 providers. Some providers did not report their gender}\\
\multicolumn{7}{l}{\footnotesize status. Some providers did not report their gender status. The facility level were selected from the listing of health facilities} \\
\multicolumn{7}{l}{\footnotesize provided by the Ministry of Health in each country. Facilties were randomly selected in each country when feasibile. The total}\\
\multicolumn{7}{l}{\footnotesize number of providers includes all providers in the sample regardless if they reported their gender or not. }\\
\end{tabular}
						\end{adjustbox}	
						\label{prov_des}
			\end{table}
		
			\begin{figure}[H] 
				\centering
				\caption{Age of Providers by  Staff Cadre} 
						\includegraphics[width=\textwidth]{"../Output/Final/Per_providers_occ"}
						\label{age_prov}
			\end{figure}
		
			\begin{figure}[H] 
				\centering
				\caption{Fraction of Female Doctors by Age} 
						\includegraphics[width=\textwidth]{"../Output/Final/Line plots/med_fem_frac"}
						\label{fem_doc}
			\end{figure}			
			
			\begin{figure}[H] 
				\centering
				\caption{Share of Clinics that have a Doctor} 
						\includegraphics[width=\textwidth]{"../Output/Final/Poverty_medone_allcountries"}
						\label{doc}
			\end{figure}
			
			\begin{figure}[H] 
				\centering
				\caption{Share of Clinics that have a Doctor by Region} 
						\includegraphics[width=\textwidth]{"../Output/Final/Poverty_medone_region"}
						\label{doc_reg}
			\end{figure}
	
	
\end{document}